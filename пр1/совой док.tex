\documentclass [12pt]{report}
\usepackage[utf8]{inputenc}
\usepackage[english, russian]{babel}
\usepackage{pscyr}
\begin{document}
\title{Курсач по автоматизированным информационным системам} % Заглавие документа
\date{\today} % Дата создания
\maketitle
\author{Aitr}


\pagenumbering {Roman}
\thispagestyle {headings}
Пробный исходный файл создан \textbf{студентом} группы 4-ИС-1-13, Преловский В.

  \begin{eqnarray}
    E &=& mc^2\\
\newline
    m &=& \frac{m_0}{\sqrt{1-\frac{v^2}{c^2}}}
 \end{eqnarray}

\newpage
\selectlanguage{russian}
\pagenumbering {asbuk}
\thispagestyle{empty}
автобиография моя вообще ни о чём не говорит. Я пишу курсачи и дипломы, верстаю простые автоматизированные рабочие мест, к примеру, для преподавателя.

Это оправдывается путём ускорения обработки и системности хранения информации. Вдобавок это очень классно и вся информация и инфографика доступна в одно нажатие. Everything is possible sometimes.

\newpage
\pagenumbering{alph}
школа и колледж.
\newsavebox{\fmbox}


Моя шарага является в реальности Иркутским Энергетическим колледжем. Это такое спо, где учат электриков и электричек. Тепловиков и тепловичек. Компьютерщиков и компьютерщичек.


 Здесь легко и доступно учиться, но не всегда понятно, что мы делаем на парах.

\end{document}