\documentclass{article}
\usepackage[utf8]{inputenc}
\usepackage[english, russian]{babel}
\usepackage{multirow}
%\usepackage{pscyr}

\begin{document}

\begin{center} \LARGE \bf{{ \textit{Форматирование текста} }} \end {center}

При работе  с абзацем \LARGE  \underline{помнить}:  \normalsize \newline
 
{}
\begin {itemize}
\item \textit{несколько пробелов \(пустых строк\) воспринимает как один \(одну\)}
\item \textit{Абзац заканчивается только тогда, когда будет вставлена пустая строка либо команда абзаца} $\backslash $par
\item \textit{По умолчанию абзац выравнивается по ширине}
\end{itemize}

\textit{Позиционирование}

	\hspace{10pt}Процедуры
\begin{center}
$\backslash $begin \{center\} ...  позволяют центрировать текст \end{center}
\begin{flushleft} выравнивать по левому краю \end{flushleft}
\begin {flushright} или по правому \end {flushright}
\begin {center} 
\linespread{1.3}
{\bf Шрифты} {\it (в 2007 г. прочитано после математики)}
\end {center}

{\it Шрифт } это \LARGE графическое   изображение букв
алфавита \tiny с относящимися к нему знаками и цифрами

Здравствуйте!
\newline

\begin{center}
	Часть № 2.
	\newline
	Таблица 1.
	\newline

	\begin{tabular}{|c|c|c|c|c|}
	\hline 
	№ & Фамилия & Имя & Отчество & Адрес \\ 
	\hline 
	1 & Иванов & Иван & Иванович & Иркутск, Энергоколледж \\ 
	\hline 
	2 & Петров & Иван & Победович & Иркутск, Площадь Сперанского, 1 \\ 
	\hline 
	3 & Сидорова & Ксения & Владимировна & Иркутск, ул. Территория прирельсового склада, д.5 \\ 
	\hline 
	\end{tabular} 
\end{center}
\newpage

%----------------------------------------------------------
% There must be created very diffcult table
\begin{flushleft}
	

\begin{tabular}{|c|c|c|c|c|c|c|c|c|c|c|}
	\hline 
	
	{\multirow{2}{*}{\textbf{№}}} & {\multirow{2}{*}{\textbf{Наименование}}} & {\multirow{2}{*}{\textbf{ед. изм}}} & {\multirow{2}{*}{\textbf{СНИП}}} & {\multirow{2}{*}{\textbf{Объем работ}}}  & \multicolumn{2}{c|}{\textbf{Норма времени}} & \multicolumn{2}{c|}{\textbf{Труд}} & \multicolumn{2}{c|}{\textbf{Машиноемкость}} \\ 
	\cline{6-11}

	&  &  &  &  & Раб & Маш & Ч час & Ч ден & Мч & Мсм \\ 
	\hline 
	\textbf{1.} & \multicolumn{9}{c}{ \textbf{Кирпичная армированная кладка  для стен под штукатурку}} &     \\ 
	\hline 
	& кладка внешней стены & куб. м & Е3-4 & 200,78 & 3,40 &   & 682,682 & 85,332 &   &   \\ 
	\hline 
	& кладка внутренних стен & Куб. м  & Е3-4  & 107,98  & 3,2  &   & 345,536 & 43,192 &  &  \\ 
	\hline 
	\textbf{2.} & \multicolumn{10}{c|}{\textbf{Дополнительные работы}} \\ 
	\hline 
	2.1 & Подмосты для стен & Кв. м & Е3-4 & 200,78 & 0,117 & 0,039 & 23.49 & 2,94 & 7,83 & 0,98 \\ 
	\hline 
	2.2 & Подача кирпичей & тыс. шт & Е3-4 & 113,01 & 0,74 & 0,37 & 83,63 & 10,45 & 41,81 & 5,23 \\ 
	\hline 
	2.3 & Подача раствора & Куб. м & Е3-4 & 64,22 & 0,24 & 0,12 & 15,41 & 1,93 & 7,7 & 0,96 \\ 
	\hline 
\end{tabular} 
\end{flushleft}

\end{document}