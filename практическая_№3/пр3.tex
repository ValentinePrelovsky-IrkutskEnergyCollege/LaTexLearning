\documentclass{article}
\usepackage[utf8]{inputenc}
\usepackage[english, russian]{babel}
\usepackage{pscyr}
\begin{document}

\begin{center} \LARGE \bf{{ \textit{Форматирование текста} }} \end {center}

При работе  с абзацем \LARGE  \underline{помнить}:  \normalsize \newline
 
{}
\begin {itemize}
\item \textit{несколько пробелов \(пустых строк\) воспринимает как один \(одну\)}
\item \textit{Абзац заканчивается только тогда, когда будет вставлена пустая строка либо команда абзаца} $\backslash $par
\item \textit{По умолчанию абзац выравнивается по ширине}
\end{itemize}

\textit{Позиционирование}

	\hspace{10pt}Процедуры
\begin{center}
$\backslash $begin \{center\} ...  позволяют центрировать текст \end{center}
\begin{flushleft} выравнивать по левому краю \end{flushleft}
\begin {flushright} или по правому \end {flushright}
\begin {center} 
\linespread{1.3}
{\bf Шрифты} {\it (в 2007 г. прочитано после математики)}
\end {center}

{\it Шрифт } это \LARGE графическое   изображение букв
алфавита \tiny с относящимися к нему знаками и цифрами


\end{document}