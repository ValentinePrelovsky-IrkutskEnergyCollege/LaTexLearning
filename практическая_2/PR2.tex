\documentclass[12pt]{article}
\usepackage[utf8]{inputenc}
\usepackage[english, russian]{babel}
\usepackage{pscyr}

 \makeatletter 
\begin{document}
\title{Name of document}
\author{A.B. Ivanov}
\maketitle
\begin{abstract}
Annotation
\end{abstract}

\newpage
\pagestyle{plain}
\renewcommand{\@oddfoot}{\hfil WOW!\hfil} 

\section{First level header}
\subsection{Second level header}
		Привет от заголовка второго уровня нашего текста. Мы изучаем профенссиональное издательство путём программ, которые были придуманы специально для этих целей.
\subsection{Third level header}
	Text of t3rd level.
\newpage
jkasjadsjkasjasjasjasjas
\thispagestyle{plain}
\makeatletter
\renewcommand{\@oddhead}{\hfill  hello world! \hfill}
\renewcommand{\@oddfoot}{{my \hfill \thepage }}

\makeatother% возвращаем знаку @ командные свойства


\newpage
\begin {thebibliography}
{lite}
\bibitem{1} Сюткин Владислав - набор математических формул.
\bibitem{2} Герберт Шилдт - полное руководство C++11
\end {thebibliography}
\end{document}