\documentclass[]{report}
\usepackage[utf8]{inputenc}
\usepackage[english, russian]{babel}
\usepackage{hyperref}

% Title Page
\title{Контроль версий кода}
\author{Преловский В.А.}

\textwidth = 5.52in	
\begin{document}
	\maketitle

	
	\begin{abstract}
		{\large 
			В данном докладе рассматриваются общие принципы системы контроля версий кода, git, и примеры работы с системой контроля версий git.
		}
	\end{abstract}
	{\large 
		\textbf{\begin{center}
				Что такое система контроля версий?
		\end{center}}
		\begin{center}
		{\huge О контроле версий}
		\\		
			\textsl{Что такое контроль версий, и зачем он вам нужен? }
		\end{center}
		
		
		\underline{Система контроля версий} (СКВ) — это система, регистрирующая изменения в одном или нескольких файлах с тем, чтобы в дальнейшем была возможность вернуться к определённым старым версиям этих файлов. Мы чаще всего используем её для исходных кодов программ, но на самом деле под версионный контроль можно поместить файлы практически любого типа.
			\newline
			
		Если вы графический или веб­дизайнер и хотели бы хранить каждую версию изображения или	макета — а этого вам наверняка хочется — то пользоваться системой контроля версий будет	очень мудрым решением. \textbf{СКВ даёт возможность возвращать отдельные файлы к прежнему виду, возвращать к прежнему состоянию весь проект, просматривать происходящие со	временем изменения, определять, кто последним вносил изменения во внезапно переставший	работать модуль, кто и когда внёс в код какую-­то ошибк}у, и многое другое. Вообще, если, пользуясь СКВ, вы всё испортите или потеряете файлы, всё можно будет легко восстановить. Вдобавок, накладные расходы за всё, что вы получаете, будут очень маленькими.
		
		Такие системы наиболее широко используются при разработке программного обеспечения для хранения исходных кодов разрабатываемой программы. Однако они могут с успехом применяться и в других областях, в которых ведётся работа с большим количеством непрерывно изменяющихся электронных документов. В частности, системы управления версиями применяются в САПР, обычно в составе систем управления данными об изделии (PDM). Управление версиями используется в инструментах конфигурационного управления (Software Configuration Management Tools).
		\\
		
		Ситуация, в которой электронный документ за время своего существования претерпевает ряд изменений, достаточно типична. При этом часто бывает важно иметь не только последнюю версию, но и несколько предыдущих. В простейшем случае можно просто хранить несколько вариантов документа, нумеруя их соответствующим образом. Такой способ неэффективен (приходится хранить несколько практически идентичных копий), требует повышенного внимания и дисциплины и часто ведёт к ошибкам, поэтому были разработаны средства для автоматизации этой работы.
		
		\textbf{Многие системы управления версиями предоставляют ряд других возможностей:}
		\\
		
		\begin{itemize}
			\item Позволяют создавать разные варианты одного документа, т. н. ветки, с общей историей изменений до точки ветвления и с разными — после неё.
			\item Дают возможность узнать, кто и когда добавил или изменил конкретный набор строк в файле.
			\item Ведут журнал изменений, в который пользователи могут записывать пояснения о том, что и почему они изменили в данной версии.
			\item Контролируют права доступа пользователей, разрешая или запрещая чтение или изменение данных, в зависимости от того, кто запрашивает это действие.
		\end{itemize}
		
		
	\begin{center}{\large{\textbf{Распределённые системы контроля версий}}}
	\end{center}		

			
		И в ситуации, когда главный сервер может "умереть", в игру вступают {\textbf{распределённые системы контроля версий (РСКВ).}
			\\
			
		В таких	системах как \textbf{\textit{Git}}, Mercurial, Bazaar или Darcs клиенты не просто выгружают последние версии	файлов, а {\textit{полностью копируют весь репозиторий}. Поэтому в случае, когда "умирает" сервер, через который шла работа, любой клиентский репозиторий может быть скопирован обратно на	сервер, чтобы восстановить базу данных. Каждый раз, когда клиент забирает свежую версию файлов, он \textbf{создаёт себе полную копию всех данных} (см. рисунок ниже).
		% вставить рисунок DVCS
		
		
		\begin {thebibliography}{10}
		\bibitem {Wikiarticle}
		{\sc Статья в Википедии},{\href{https://ru.wikipedia.org/wiki/Система_управления_версиями}{о системах контроля версий кода}}
		
		\bibitem {OnlineBook}
		{\sc Книга Скотта Шакона},{\href{https://git-scm.com/book/en/v2}{Онлайн книга Скотта Шакона Pro Git}}
		\end {thebibliography}
	}
\end{document}    